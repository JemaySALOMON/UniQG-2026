\documentclass[c]{beamer} % use [t] to top-justified body text by default
% \documentclass[c,handout]{beamer}
\usepackage{graphicx}
\hypersetup{colorlinks, linkcolor=black, urlcolor=gray}
\usepackage{amsmath}
\usepackage{bm} % to have mathematical symbols in bold
\usepackage{multirow}
\usepackage{tikz}
% \usepackage[francais]{babel}
\usepackage[english]{babel}
\usepackage[T1]{fontenc} 
\usepackage[utf8]{inputenc}
\usepackage{tcolorbox}
\usepackage{multicol}
\usepackage{xcolor}
\usepackage{hyperref}

% https://tex.stackexchange.com/a/82558
\usepackage[absolute,overlay]{textpos}
% \usepackage[texcoord,grid,gridunit=mm,gridcolor=red!10,subgridcolor=green!10]{eso-pic}

\graphicspath{{./figures/}}

% -----------------------------------------------------------------------------
\setbeamertemplate{navigation symbols}{}
\setbeamercolor{alerted text}{fg=purple}
\setbeamertemplate{caption}[numbered]
\setbeamerfont{caption}{size=\scriptsize}
\setbeameroption{hide notes}
% \setbeameroption{show notes}
\setbeamertemplate{note page}[plain]

\setbeamertemplate{frametitle}{
  \vspace{0.2cm}
  \makebox[\linewidth][c]{
    \parbox{0.9\linewidth}{\centering
      \bfseries\insertframetitle
    }
  }
}

\setbeamertemplate{footline}
{
  \leavevmode
  \hbox{
    \hspace*{-0.06cm}
    \begin{beamercolorbox}[wd=.2\paperwidth,ht=2.25ex,dp=1ex,center]{author in head/foot}
      \usebeamerfont{author in head/foot}\insertshortauthor \hspace*{1em} \insertshortinstitute
    \end{beamercolorbox}
    \begin{beamercolorbox}[wd=.50\paperwidth,ht=2.25ex,dp=1ex,center]{section in head/foot}
      \usebeamerfont{section in head/foot}\insertshorttitle
    \end{beamercolorbox}
    \begin{beamercolorbox}[wd=.27\paperwidth,ht=2.25ex,dp=1ex,right]{section in head/foot}%
      \usebeamerfont{section in head/foot}\insertshortdate{}\hspace*{2em}
      \insertframenumber{} / \inserttotalframenumber\hspace*{2ex}
    \end{beamercolorbox}
  }
  \vskip0pt
}

\AtBeginSection[]
{
  \begin{frame}
    \frametitle{Outline}
    \addtocounter{framenumber}{-1}
    \tableofcontents[currentsection]
  \end{frame}
}

\AtBeginSubsection[]
{
  \begin{frame}
    \frametitle{Outline}
    \addtocounter{framenumber}{-1}
    \tableofcontents[currentsubsection,hideothersubsections,sectionstyle=show/shaded,subsectionstyle=show/shaded]
  \end{frame}
}
% -----------------------------------------------------------------------------


%%%%%%%%%%%%%%%%%%%%%%%%%%%%
%% Beginning of the document
%%%%%%%%%%%%%%%%%%%%%%%%%%%%
\begin{document}

\title[Génét. Quanti., UniQ ]{Cours 6 : Interactions G×E}
\author[J. Salomon]{Jemay Salomon}
\institute[]{\small UMR GQE Le Moulon \\ Université Paris–Saclay, INRAE, CNRS, AgroParisTech}
\date{Jan. 06, 2026}

%% Title page
\begin{frame}[plain,t]
\noindent
%\includegraphics[height=0.7cm]{figures/logo-gqe-le-moulon.png}\hfill
%\includegraphics[height=0.8cm]{figures/logos_mobidiv.jpeg}

\vfill
\begin{center}
  %PLANTCOM meeting, Dijon

  \bigskip

  \bigskip
  
  {\fontsize{14pt}{16pt}\selectfont\bfseries \inserttitle}\\[0.8cm] 
  {\large\insertauthor}\\[0.4cm] 
  {\normalsize\insertinstitute}
\end{center}
\vfill

\noindent
\includegraphics[width=2cm]{figures/logo_faculte_sciences.png}\hfill
\includegraphics[width=0.8cm]{figures/logo-inrae-fond-blanc.png}\hfill
\includegraphics[width=0.5cm]{figures/logo_CNRS_biologie.png}\hfill
\includegraphics[width=1.5cm]{figures/logo_agroparistech.png}
\end{frame}



\begin{frame}
  \frametitle{Outline}
  \tableofcontents
\end{frame}


%%%%%%%%%%%%%%%%%%%%%%%%%%% 
%% --Begin-Document --%
%%%%%%%%%%%%%%%%%%%%%%%%%%% 

\section{Contexte}
\begin{frame}
\frametitle{Contexte}
\begin{columns}
\begin{column}{0.55\textwidth}
\begin{itemize}
  \item L’interaction génotype \(\times\) environnement (G\(\times\)E) complique les décisions de sélection des sélectionneurs
  \item Les individus (génotypes) réagissent différemment aux variations de leur environnement
  \item Les sélectionneurs s’intéressent aux génotypes bien adaptés à leur population cible d’environnements (TPE)
  \item Par exemple : des génotypes à haut rendement et stables à travers les années et les localisations
\end{itemize}
\end{column}

\begin{column}{0.45\textwidth}
\centering
\textbf{Env.1}\\
\includegraphics[width=0.9\textwidth]{figures/env1.png}\\[0.4cm]
\textbf{Env.2}\\
\includegraphics[width=0.9\textwidth]{figures/env2.png}
\end{column}
\end{columns}
\end{frame}

\section{Interactions Genotypes x Environnements}
\begin{frame}
\frametitle{Interactions GXE}
\begin{columns}
\begin{column}{0.6\textwidth}
\begin{itemize}
  \item La similarité entre environnements est souvent évaluée à l’aide des \textbf{corrélations génétiques}
  \item Corrélation = 1 : accord parfait des classements des génotypes entre environnements
  \item Corrélation = 0 : absence de similarité dans les classements entre environnements
  \item Corrélation = -1 : inversion complète des classements des génotypes entre environnements
\end{itemize}
\end{column}

% Colonne droite : espace pour image
\begin{column}{0.4\textwidth}
\centering
\vspace{1cm}
\fbox{\parbox[c][4cm][c]{0.9\textwidth}{
\centering
\includegraphics[width=0.9\textwidth]{figures/env_cor.png}
}}
\end{column}
\end{columns}
\end{frame}

\begin{frame}
\frametitle{Types d'Interactions G$\times$E}
\begin{columns}
  \begin{column}{0.5\textwidth}
    \centering
    \includegraphics[width=\textwidth]{figures/types_1.png}
  \end{column}

  \begin{column}{0.5\textwidth}
    \centering
    \includegraphics[width=\textwidth]{figures/types_2.png}
  \end{column}
\end{columns}
\end{frame}

\begin{frame}
\frametitle{Types d'interactions G$\times$E}
\centering
\includegraphics[width=\textwidth]{figures/both_GE.png}
\vspace{0.2cm}
{\footnotesize Source : EUCARPIA 2025 Biometrics Plant Breeding — adapté de Tolhust}
\end{frame}


\section{MET-TPE}
\subsection{TPE-Target Population Environnement}
\begin{frame}
\frametitle{TPE-Target Population Environnement}
\centering
\includegraphics[width=\textwidth]{figures/TPE.png}
\vspace{0.2cm}
{\footnotesize Source : EUCARPIA 2025 Biometrics Plant Breeding — adapté de Tolhust}
\end{frame}

\begin{frame}
\frametitle{TPE-Target Population Environnement}
\centering
\includegraphics[width=\textwidth]{figures/MET_TPE.png}
\vspace{0.2cm}
{\footnotesize Cooper et al. (2023); Bančič et al. (2024)}
\end{frame}

\begin{frame}
\frametitle{Comment gérer l’interaction G$\times$E ?}
\begin{itemize}
  \item Historiquement, l’interaction G$\times$E a été abordée de trois manières principales :
  \begin{enumerate}
    \item \textbf{Ignorer la G$\times$E} en sélectionnant les génotypes les plus performants en moyenne
    \item \textbf{Réduire la G$\times$E} en regroupant des environnements similaires et en sélectionnant au sein de chaque groupe
    \item \textbf{Exploiter la G$\times$E} en sélectionnant des individus performants en moyenne et stables (adaptabilité)
  \end{enumerate}
  \item Une combinaison des approches \textbf{(2)} et \textbf{(3)} peut également être envisagée
\end{itemize}
\end{frame}

\subsection{MET-Multi-environment trial}
\begin{frame}
\frametitle{MET-Multi-environment trial}
\centering
\only<1>{
\includegraphics[width=\textwidth]{figures/MET.png}
\vspace{0.2cm}
{\footnotesize Source : EUCARPIA 2025 Biometrics Plant Breeding — adapté de Tolhust}
}

\only<2>{
\includegraphics[width=\textwidth]{figures/breeding.png}
\vspace{0.2cm}
{\footnotesize Source : EUCARPIA 2025 Biometrics Plant Breeding — adapté de Tolhust}
}
\end{frame}


\section{Modèles Linéaires pour MET}
\begin{frame}
\frametitle{RCBD}

\[
y_{ij} = \mu + g_i + b_j + \epsilon_{ij}
\]

\begin{columns}
\begin{column}{0.55\textwidth}
\begin{itemize}
  \item simple à contruire
  \item Complet/equilibré
  \item Facile à résoudre
\end{itemize}
\end{column}

\begin{column}{0.45\textwidth}
\centering
\includegraphics[width=\textwidth]{figures/rcbd.png}
\end{column}
\end{columns}
\end{frame}


\begin{frame}
\frametitle{Extension aux MET}

\[
y_{ijm} = \mu +  \alpha_m + g_{im} + b_{jm} + ..Interactions ...+ \epsilon_{ijm}
\]

\begin{itemize}
  \item y : phenotype (ex. rendement)
  \item $\mu$ intercept
  \item $ \alpha_m$ : effet environnement
   \item $ b_{jm}$ : effet block
   \item $\epsilon_{ijm}$ : residus
\end{itemize}
\end{frame}





\begin{frame}
\frametitle{Décomposition des interactions G$\times$E}
\centering
\only<1>{
\textbf{Métriques couramment utilisées :}

\begin{itemize}
  \item Analyse de la variance (ANOVA)
  \item Modèles de régression  
        (Finlay \& Wilkinson, 1963 ; Eberhart \& Russell, 1966)
  \item Écovalence de Wricke (1962)
  \item Variance de Shukla (1972)
  \item Indice de Kang (YSi) / rKang (1991)
  \item Indice de supériorité (Lin \& Binns, 1988)
  \item Rendement moyen
\end{itemize}
}

\only<2>{
\includegraphics[width=\textwidth]{figures/part_GE.png}

\vspace{0.2cm}
{\footnotesize Mémoire de Master 2 (Salomon, 2023)}
}
\end{frame}


% After your last numbered slide
\appendix
\newcounter{finalframe}
\setcounter{finalframe}{\value{framenumber}}
\setcounter{framenumber}{\value{finalframe}}

\end{document}

%%% Local Variables:
%%% mode: latex
%%% TeX-master: t
%%% End:
