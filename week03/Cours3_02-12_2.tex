\documentclass[c]{beamer} % use [t] to top-justified body text by default
% \documentclass[c,handout]{beamer}
\usepackage{graphicx}
\usepackage{hyperref}
\hypersetup{colorlinks, linkcolor=black, urlcolor=gray}
\usepackage{amsmath}
\usepackage{bm} % to have mathematical symbols in bold
\usepackage{multirow}
\usepackage{tikz}
% \usepackage[francais]{babel}
\usepackage[english]{babel}
\usepackage[T1]{fontenc} 
\usepackage[utf8]{inputenc}
\usepackage{tcolorbox}
\usepackage{multicol}
\usepackage{xcolor}

% https://tex.stackexchange.com/a/82558
\usepackage[absolute,overlay]{textpos}
% \usepackage[texcoord,grid,gridunit=mm,gridcolor=red!10,subgridcolor=green!10]{eso-pic}

\graphicspath{{./figures/}}

% -----------------------------------------------------------------------------
\setbeamertemplate{navigation symbols}{}
\setbeamercolor{alerted text}{fg=purple}
\setbeamertemplate{caption}[numbered]
\setbeamerfont{caption}{size=\scriptsize}
\setbeameroption{hide notes}
% \setbeameroption{show notes}
\setbeamertemplate{note page}[plain]

\setbeamertemplate{frametitle}{
  \vspace{0.2cm}
  \makebox[\linewidth][c]{
    \parbox{0.9\linewidth}{\centering
      \bfseries\insertframetitle
    }
  }
}

\setbeamertemplate{footline}
{
  \leavevmode
  \hbox{
    \hspace*{-0.06cm}
    \begin{beamercolorbox}[wd=.2\paperwidth,ht=2.25ex,dp=1ex,center]{author in head/foot}
      \usebeamerfont{author in head/foot}\insertshortauthor \hspace*{1em} \insertshortinstitute
    \end{beamercolorbox}
    \begin{beamercolorbox}[wd=.50\paperwidth,ht=2.25ex,dp=1ex,center]{section in head/foot}
      \usebeamerfont{section in head/foot}\insertshorttitle
    \end{beamercolorbox}
    \begin{beamercolorbox}[wd=.27\paperwidth,ht=2.25ex,dp=1ex,right]{section in head/foot}%
      \usebeamerfont{section in head/foot}\insertshortdate{}\hspace*{2em}
      \insertframenumber{} / \inserttotalframenumber\hspace*{2ex}
    \end{beamercolorbox}
  }
  \vskip0pt
}

\AtBeginSection[]
{
  \begin{frame}
    \frametitle{Outline}
    \addtocounter{framenumber}{-1}
    \tableofcontents[currentsection]
  \end{frame}
}

\AtBeginSubsection[]
{
  \begin{frame}
    \frametitle{Outline}
    \addtocounter{framenumber}{-1}
    \tableofcontents[currentsubsection,hideothersubsections,sectionstyle=show/shaded,subsectionstyle=show/shaded]
  \end{frame}
}
% -----------------------------------------------------------------------------


%%%%%%%%%%%%%%%%%%%%%%%%%%%%
%% Beginning of the document
%%%%%%%%%%%%%%%%%%%%%%%%%%%%
\begin{document}

\title[Génét. Quanti., UniQ ]{Cours 3 – Covariances entre apparentés}
\author[J. Salomon]{Jemay Salomon}
\institute[]{\small UMR GQE Le Moulon \\ Université Paris–Saclay, INRAE, CNRS, AgroParisTech}
\date{Nov. 25, 2025}

%% Title page
\begin{frame}[plain,t]
\noindent
%\includegraphics[height=0.7cm]{figures/logo-gqe-le-moulon.png}\hfill
%\includegraphics[height=0.8cm]{figures/logos_mobidiv.jpeg}

\vfill
\begin{center}
  %PLANTCOM meeting, Dijon

  \bigskip

  \bigskip
  
  {\fontsize{14pt}{16pt}\selectfont\bfseries \inserttitle}\\[0.8cm] 
  {\large\insertauthor}\\[0.4cm] 
  {\normalsize\insertinstitute}
\end{center}
\vfill

\noindent
\includegraphics[width=2cm]{figures/logo_faculte_sciences.png}\hfill
\includegraphics[width=0.8cm]{figures/logo-inrae-fond-blanc.png}\hfill
\includegraphics[width=0.5cm]{figures/logo_CNRS_biologie.png}\hfill
\includegraphics[width=1.5cm]{figures/logo_agroparistech.png}
\end{frame}



\begin{frame}
  \frametitle{Outline}
  \tableofcontents
\end{frame}


%%%%%%%%%%%%%%%%%%%%%%%%%%% 
%% --Begin-Document --%
%%%%%%%%%%%%%%%%%%%%%%%%%%% 
\section{Hypothèses du modèle}
\begin{frame}
 \frametitle{Covariances entre apparentés : hypothèses et notations}

\only<1>{
\textbf{Population de référence}
\begin{itemize}
  \item Population très grande, panmictique, sans sélection ni migration.
  \item Les allèles sont tirés au hasard selon leurs fréquences alléliques.
\end{itemize}

\textbf{Deux situations entre individus}
\begin{itemize}
  \item \textbf{Apparentés} : ils partagent un ancêtre récent $\Rightarrow$ possibilité d’allèles identiques par descendance.
  \item \textbf{Non apparentés} : probabilité d’identité par descendance supposée nulle.
\end{itemize}
}

\only<2>{

\medskip
Soient deux individus au même locus :
\[
G_1 = (i, j)\qquad G_2 = (i', j')
\]

On définit l'\textbf{identité par descendance} (IBD) :
\[
i \equiv i' \quad \Longleftrightarrow \quad \text{les deux allèles proviennent d'un même allèle ancestral.}
\]

Probabilité IBD: \textbf{coefficient de simple parenté (Malécot,1948)} :
\[
\varphi_A = P(i \equiv i')
\]

\textbf{Propriété clé :}
\[
E(\alpha_i \alpha_{i'}) =
\begin{cases}
E(\alpha_i^2)=V(\alpha), & \text{si } i \equiv i' ,\\[4pt]
0, & \text{si } i \not\equiv i' .
\end{cases}
\]

Donc, en moyenne,
\[
E(\alpha_i \alpha_{i'}) \;=\; \varphi_A\, V(\alpha).
\]
}
\end{frame}


\begin{frame}
\frametitle{Modèle génétique à un locus (Fisher)}
\only<1>{
Pour un individu de génotype $(i,j)$ :
\[
Y_{ij} = \mu + \alpha_i + \alpha_j + \beta_{ij}
\]

De même pour un second individu $(i',j')$ :
\[
Y_{i'j'} = \mu + \alpha_{i'} + \alpha_{j'} + \beta_{i'j'}
\]

Où :
\begin{itemize}
\item $\mu$ : moyenne de la population
\item $\alpha_i$ : effet additif de l’allèle $i$
\item $\beta_{ij}$ : effet de dominance (interaction entre $i$ et $j$)
\end{itemize}
}

\only<2>{
\textbf{Par construction du modèle :}
\[
E(\alpha_i) = 0,\qquad E(\beta_{ij}) = 0
\]
\[
\operatorname{Cov}(\alpha, \beta) = 0
\]

Et en panmixie (absence d’IBD) :
\[
\operatorname{Cov}(\beta_{ij}, \beta_{i'j'}) = 0
\quad \text{si pas d’identité par descendance}
\]

Notre objectif : calculer
\[
\operatorname{Cov}(Y_{ij},\, Y_{i'j'})
\]
}
\end{frame}

\begin{frame}

\frametitle{Développement détaillé de $\operatorname{Cov}(Y_{ij}, Y_{i'j'})$}

\only<1>{
\[
\operatorname{Cov}(Y_{ij}, Y_{i'j'})
= \operatorname{Cov}(\alpha_i + \alpha_j + \beta_{ij},\; \alpha_{i'} + \alpha_{j'} + \beta_{i'j'})
\]

Développement complet :

\[
= E(\alpha_i \alpha_{i'})
+ E(\alpha_i \alpha_{j'})
+ E(\alpha_j \alpha_{i'})
+ E(\alpha_j \alpha_{j'})
+ E(\beta_{ij} \beta_{i'j'})
\]
\[
+ \operatorname{Cov}(\alpha_i,\beta_{i'j'})
+ \operatorname{Cov}(\alpha_j,\beta_{i'j'})
+ \operatorname{Cov}(\alpha_{i'},\beta_{ij})
+ \operatorname{Cov}(\alpha_{j'},\beta_{ij})
\]

Par hypothèse de panmixie :
\[
\operatorname{Cov}(\alpha,\beta)=0
\]
donc les 4 derniers termes sont nuls.

Il reste donc :
\[
\operatorname{Cov}(Y_{ij}, Y_{i'j'})
= E(\alpha_i \alpha_{i'})
+ E(\alpha_i \alpha_{j'})
+ E(\alpha_j \alpha_{i'})
+ E(\alpha_j \alpha_{j'})
+ E(\beta_{ij} \beta_{i'j'})
\]

}

\only<2>{
Si les individus sont apparentés :
\[
P(i \equiv i')
= P(i \equiv j')
= P(j \equiv i')
= P(j \equiv j')
= \varphi_A
\]

et ainsi :
\[
E(\alpha_i \alpha_{i'})
= E(\alpha_i \alpha_{j'})
= E(\alpha_j \alpha_{i'})
= E(\alpha_j \alpha_{j'})
= \varphi_A V(\alpha)
\]

Donc :
\[
\boxed{
\operatorname{Cov}(Y_{ij}, Y_{i'j'})
= 4 \varphi_A V(\alpha)
+ E(\beta_{ij} \beta_{i'j'})
}
\]
}
\end{frame}



\begin{frame}
\frametitle{Développement détaillé de $\operatorname{Cov}(Y_{ij}, Y_{i'j'})$}

\only<1>{
\textbf{Forme générale :}

\[
\boxed{
\operatorname{Cov}(Y_{ij}, Y_{i'j'})
= 4 \, \varphi_A \, \operatorname{Var}(\alpha)
+ \varphi_D \, \operatorname{Var}(\beta)
}
\]

où :
\begin{itemize}
\item $\operatorname{Var}(\alpha)$ : variance additive par allèle
\item $\operatorname{Var}(\beta)$ : variance de dominance
\item $\varphi_A$ : coefficient de parenté (Malécot)
\item $\varphi_D$ : probabilité d’identité par descendance des deux couples d’allèles
\end{itemize}
}

\only<2>{
\textbf{Lien avec la variance additive totale :}

\[
V_A = 2 \, \operatorname{Var}(\alpha)
\]

\[
4 \, \varphi_A \, \operatorname{Var}(\alpha) = 2 \, \varphi_A \, V_A
\]

\textbf{Covariance :}

\[
\operatorname{Cov}(X, Y)
= 2 \, \varphi_A \, V_A
+ \varphi_D \, V_D
\]

}

\end{frame}

\begin{frame}
\frametitle{Exemple : Covariance Parent-Enfant}

\only<1>{
\textbf{Cas 1 : Parent hétérozygote, population panmictique}

Parent : $X = A_i A_j$, $i \neq j$  
Autre parent : $A_{i'}A_{j'}$ non apparentés

Enfant : $Y$ peut être $A_iA_{i'}$, $A_iA_{j'}$, $A_jA_{i'}$, $A_jA_{j'}$  

\[
\begin{cases}
\text{Si l’allèle transmis est } A_{i'} \text{ ou } A_{j'} : P(i \equiv i') = 0 \\
\text{Si l’allèle transmis est } A_i \text{ ou } A_j : P(i \equiv i') = 1/2
\end{cases}
\]

Au total : 
\[
\varphi_A = \frac{1}{2} \times \frac{1}{2} = \frac{1}{4}, \qquad \varphi_D = 0
\]

\[
\operatorname{Cov}(X,Y) = 2 \, \varphi_A \, V_A = 2 \times \frac{1}{4} V_A = \frac{1}{2} V_A
\]
}

\only<2>{
\textbf{Cas 2 : Parent homozygote}

Parent : $X = A_i A_i$  
Autre parent : $A_j A_j$ (différent)  

L’allèle transmis est forcément $i$, probabilité $1/2$ chez un diploïde  

\[
\varphi_A = \frac{1}{2}, \qquad \varphi_D = 0
\]

\[
\operatorname{Cov}(X,Y) = 2 \, \varphi_A \, V_A = 2 \times \frac{1}{2} V_A = V_A
\]
}

\only<3>{
\textbf{Cas 3 : Autofécondation d’un parent hétérozygote}

Parent : $X = A_i A_j$, $i \neq j$  

Descendant issu d’autofécondation : $Y$  

\[
\varphi_A = \frac{1}{2}, \qquad \varphi_D = \frac{1}{2}
\]

\[
\operatorname{Cov}(X,Y) = 2 \, \varphi_A \, V_A + \varphi_D \, V_D
= V_A + \frac{1}{2} V_D
\]
}
\end{frame}


\begin{frame}
\frametitle{Exemples de covariance entre apparentés}

\begin{center}
\begin{tabular}{|l|c|c|c|}
\hline
\textbf{Famille} & $\varphi_A$ & $\varphi_D$ & $\operatorname{Cov}$ \\
\hline
Plein-frères & 1/4 & 1/4 & 1/2 $V_A$ + 1/4 $V_D$ \\
Demi-frères & 1/8 & 0 & 1/4 $V_A$ \\
Frères jumeaux & 1/2 & 1/2 & $V_A + V_D$ \\
\hline
\end{tabular}
\end{center}

\end{frame}




%%%%%%%%%%%%%%%%%%%%%%%%%%% 
%% --section --%
%%%%%%%%%%%%%%%%%%%%%%%%%%% 
\section{Utilisation des covariances pour estimer VA, VD}

\begin{frame}{Estimation de $V_A$ par régression parent-enfant}

\only<1>{
\textbf{Principe :} comparer le phénotype d’un enfant $Y$ avec celui de son parent $X$.

\[
Y_i = b \, X_i + \varepsilon_i
\]

où $\varepsilon_i \sim N(0, \sigma_\varepsilon^2)$ et indépendant de $X_i$.
}

\only<2>{
\textbf{Lien avec la covariance :}

\[
b = \frac{\operatorname{Cov}(Y_i, X_i)}{V(X_i)}
\]

Or pour un parent et son enfant : 
\[
\operatorname{Cov}(Y_i, X_i) = \frac{1}{2} V_A
\]

Donc :
\[
V_A = 2 \, b \, V(X_i)
\]
}

\only<3>{
\textbf{Remarque :} plusieurs descendants par parent

\begin{itemize}
\item Si plusieurs descendants, utiliser la moyenne des enfants comme $Y_i$
\item Si on utilise la moyenne des deux parents, la variance parentale devient $V(X_i)/2$
\item Le coefficient de régression $b$ devient donc $V_A/V_P$ (et non $1/2 V_A/V_P$)
\item Cela réduit l’erreur d’estimation de $b$
\end{itemize}
}

\end{frame}


\begin{frame}{Estimation de $V_A$ et $V_D$ par ANOVA : demi-frères}
\only<1>{
\textbf{Principe :} utiliser des familles homogènes et indépendantes

\[
Y_{ij} = \mu + f_i + \varepsilon_{ij}
\]

où : 
\begin{itemize}
\item $i$ = indice de la famille
\item $j$ = individu dans la famille
\item $f_i \sim N(0, \sigma_f^2)$ : effet inter-familles
\item $\varepsilon_{ij} \sim N(0, \sigma_\varepsilon^2)$ : effet intra-famille
\end{itemize}
}

\only<2>{
\textbf{Covariance intra-famille :}

Pour deux individus de la même famille :

\[
\operatorname{Cov}(Y_{ij}, Y_{ik}) = \operatorname{Cov}(f_i + \varepsilon_{ij}, f_i + \varepsilon_{ik}) = \sigma_f^2
\]

\textbf{Interprétation :} la variance inter-familles $\sigma_f^2$ correspond à la covariance entre demi-frères.

\[
\text{Donc : } V_A = 4 \, \sigma_f^2
\]
}

\only<3>{
\textbf{Extension : plein frères et designs hiérarchiques}

\begin{itemize}
\item Plein frères : covariance intra-famille = $1/2 V_A + 1/4 V_D$
\item Hiérarchie full-sib/half-sib : 
\[
Y_{ijk} = \mu + s_i + d_{ij} + \varepsilon_{ijk}
\]
où $s_i$ = effet père, $d_{ij}$ = effet mère, $\varepsilon_{ijk}$ = résidu
\item Covariances : 
\[
\operatorname{Cov}(\text{plein-frères}) = \sigma_s^2 + \sigma_d^2 = 1/2 V_A + 1/4 V_D
\]
\[
\operatorname{Cov}(\text{demi-frères}) = \sigma_s^2 = 1/4 V_A
\]
\item On peut donc estimer $V_A = 4 \sigma_s^2$ et $V_D = 4\sigma_d^2 - 4\sigma_s^2$
\end{itemize}
}
\end{frame}


\begin{frame}{Estimation via pedigrees complexes}

\textbf{Principe :}

\begin{itemize}
\item Les pedigrees peuvent inclure des relations variées : parent-enfant, pleins frères, oncle-nièce, etc.
\item Exemple simplifié : 
\begin{itemize}
  \item A > B (B est enfant de A) 
  \item C — D (C et D sont pleins frères)
\end{itemize}
\item Les covariances entre individus dépendent de leur degré d’apparentement.
\item Les relations complexes ne peuvent pas être traitées par ANOVA classique.
\item On utilise alors un \textbf{modèle mixte} (ou « modèle animal ») qui permet d’estimer VA et VD à partir de tout type de pedigree.
\item Cette approche peut être étendue aux données génomiques pour construire automatiquement les matrices d’apparentement.
\end{itemize}

\end{frame}




\begin{frame}{Estimation via données génomiques (VanRaden, 2008)}

\textbf{Principe :}

\begin{itemize}
\item On peut utiliser les génotypes des individus pour estimer leur degré d’apparentement.
\item La matrice $\mathbf{A}$ (apparentement additive) et $\mathbf{D}$ (dominance) peut être construite à partir des marqueurs SNP.
\item La covariance génétique entre individus s’écrit toujours :
\item Permet d’estimer $V_A$ et $V_D$ même sans pédigrée complet, directement à partir des données génomiques.
\end{itemize}

\end{frame}


\begin{frame}{Pourquoi prendre en compte les relations entre individus ?}

\begin{itemize}
\item \textbf{Précision :}  Ameliore la précision sur les estimations des valeurs génétiques
\item \textbf{Améliorer la précision :} tenir compte des liens familiaux augmente la puissance des analyses, par exemple en GWAS ou en sélection génomique.  
\item \textbf{Flexibilité :} permet d’intégrer des pedigrees complexes ou des données génomiques pour modéliser l’ensemble des relations dans la population.
\end{itemize}

\end{frame}

\section{Application}

\begin{frame}{Exemple d'application}
    \begin{center}
      \includegraphics[width=1\textwidth,height=1\textheight,keepaspectratio=true]{figures/cor_bv.png}
    \end{center}
\end{frame}



% After your last numbered slide
\appendix
\newcounter{finalframe}
\setcounter{finalframe}{\value{framenumber}}

\setcounter{framenumber}{\value{finalframe}}

\end{document}

%%% Local Variables:
%%% mode: latex
%%% TeX-master: t
%%% End:
