\documentclass[c]{beamer} % use [t] to top-justified body text by default
% \documentclass[c,handout]{beamer}
\usepackage{graphicx}
\hypersetup{colorlinks, linkcolor=black, urlcolor=gray}
\usepackage{amsmath}
\usepackage{bm} % to have mathematical symbols in bold
\usepackage{multirow}
\usepackage{tikz}
% \usepackage[francais]{babel}
\usepackage[english]{babel}
\usepackage[T1]{fontenc} 
\usepackage[utf8]{inputenc}
\usepackage{tcolorbox}
\usepackage{multicol}
\usepackage{xcolor}
\usepackage{hyperref}

% https://tex.stackexchange.com/a/82558
\usepackage[absolute,overlay]{textpos}
% \usepackage[texcoord,grid,gridunit=mm,gridcolor=red!10,subgridcolor=green!10]{eso-pic}

\graphicspath{{./figures/}}

% -----------------------------------------------------------------------------
\setbeamertemplate{navigation symbols}{}
\setbeamercolor{alerted text}{fg=purple}
\setbeamertemplate{caption}[numbered]
\setbeamerfont{caption}{size=\scriptsize}
\setbeameroption{hide notes}
% \setbeameroption{show notes}
\setbeamertemplate{note page}[plain]

\setbeamertemplate{frametitle}{
  \vspace{0.2cm}
  \makebox[\linewidth][c]{
    \parbox{0.9\linewidth}{\centering
      \bfseries\insertframetitle
    }
  }
}

\setbeamertemplate{footline}
{
  \leavevmode
  \hbox{
    \hspace*{-0.06cm}
    \begin{beamercolorbox}[wd=.2\paperwidth,ht=2.25ex,dp=1ex,center]{author in head/foot}
      \usebeamerfont{author in head/foot}\insertshortauthor \hspace*{1em} \insertshortinstitute
    \end{beamercolorbox}
    \begin{beamercolorbox}[wd=.50\paperwidth,ht=2.25ex,dp=1ex,center]{section in head/foot}
      \usebeamerfont{section in head/foot}\insertshorttitle
    \end{beamercolorbox}
    \begin{beamercolorbox}[wd=.27\paperwidth,ht=2.25ex,dp=1ex,right]{section in head/foot}%
      \usebeamerfont{section in head/foot}\insertshortdate{}\hspace*{2em}
      \insertframenumber{} / \inserttotalframenumber\hspace*{2ex}
    \end{beamercolorbox}
  }
  \vskip0pt
}

\AtBeginSection[]
{
  \begin{frame}
    \frametitle{Outline}
    \addtocounter{framenumber}{-1}
    \tableofcontents[currentsection]
  \end{frame}
}

\AtBeginSubsection[]
{
  \begin{frame}
    \frametitle{Outline}
    \addtocounter{framenumber}{-1}
    \tableofcontents[currentsubsection,hideothersubsections,sectionstyle=show/shaded,subsectionstyle=show/shaded]
  \end{frame}
}
% -----------------------------------------------------------------------------


%%%%%%%%%%%%%%%%%%%%%%%%%%%%
%% Beginning of the document
%%%%%%%%%%%%%%%%%%%%%%%%%%%%
\begin{document}

\title[Génét. Quanti., UniQ ]{Réponse à la sélection}
\author[J. Salomon]{Jemay Salomon}
\institute[]{\small UMR GQE Le Moulon \\ Université Paris–Saclay, INRAE, CNRS, AgroParisTech}
\date{Dec. 16, 2025}

%% Title page
\begin{frame}[plain,t]
\noindent
%\includegraphics[height=0.7cm]{figures/logo-gqe-le-moulon.png}\hfill
%\includegraphics[height=0.8cm]{figures/logos_mobidiv.jpeg}

\vfill
\begin{center}
  %PLANTCOM meeting, Dijon

  \bigskip

  \bigskip
  
  {\fontsize{14pt}{16pt}\selectfont\bfseries \inserttitle}\\[0.8cm] 
  {\large\insertauthor}\\[0.4cm] 
  {\normalsize\insertinstitute}
\end{center}
\vfill

\noindent
\includegraphics[width=2cm]{figures/logo_faculte_sciences.png}\hfill
\includegraphics[width=0.8cm]{figures/logo-inrae-fond-blanc.png}\hfill
\includegraphics[width=0.5cm]{figures/logo_CNRS_biologie.png}\hfill
\includegraphics[width=1.5cm]{figures/logo_agroparistech.png}
\end{frame}



\begin{frame}
  \frametitle{Outline}
  \tableofcontents
\end{frame}


%%%%%%%%%%%%%%%%%%%%%%%%%%% 
%% --Begin-Document --%
%%%%%%%%%%%%%%%%%%%%%%%%%%% 


\section{Motivation}
\begin{frame}{Pourquoi étudier la réponse à la sélection ?}
\textbf{Sélection artificielle et évolution}

\begin{itemize}
  \item En \textbf{amélioration des plantes et des animaux}, la sélection vise à
  \begin{itemize}
    \item améliorer la valeur moyenne d’une population,
    \item orienter l’évolution des caractères dans une direction désirée,
    \item en choisissant les parents de la génération suivante,
    \item et en contrôlant les schémas de reproduction (autofécondation, allofécondation).
  \end{itemize}

  \vspace{0.3cm}
  \item En \textbf{populations naturelles} ou en \textbf{pré-breeding}, l’objectif est de
  \begin{itemize}
    \item comprendre comment la sélection modifie les caractères,
    \item décrire les trajectoires évolutives,
    \item identifier les traits sur lesquels la sélection agit le plus fortement.
  \end{itemize}
\end{itemize}

\vspace{0.2cm}
\textit{ Lynch \& Walsh (1998), Gallais (1990)}
\end{frame}

\begin{frame}{Prédire l’effet de la sélection}
\textbf{Un enjeu central en génétique quantitative}

\begin{itemize}
  \item Un problème fondamental est de \textbf{prédire la valeur des descendants}
  issus des individus sélectionnés.
  
  \vspace{0.2cm}
  \item À partir du cadre statistique vu dans les cours précédents, on cherche à prédire
  \begin{itemize}
    \item le changement de moyenne phénotypique,
    \item la réponse de la population à la sélection.
  \end{itemize}

  \vspace{0.2cm}
  \item Ces prédictions permettent :
  \begin{itemize}
    \item d’optimiser les schémas de sélection en amélioration,
    \item de comprendre les facteurs évolutifs gouvernant l’évolution des caractères,
    \item y compris l’évolution de la fitness elle-même.
  \end{itemize}
\end{itemize}
\end{frame}

\section{Réponse à la sélection}
\subsection{Modélisation théorique}
\begin{frame}{Modélisation théorique}
\begin{itemize}
  \item On considère \textbf{un seul caractère} mesuré dans une population
  composée de $I$ génotypes distincts.
  
  \item À chaque génotype $i \in \{1,\ldots,I\}$ est associée une
  \textbf{valeur génotypique} $g_i$.

  \vspace{0.2cm}
  \item Dans la théorie classique de la génétique quantitative :
  \[
  g_i = a_i + d_i + \zeta_i
  \]

  \item où :
  \begin{itemize}
    \item $a_i$ : effet \textbf{additif} (valeur transmissible),
    \item $d_i$ : effet de \textbf{dominance},
    \item $\zeta_i$ : effet d’\textbf{épistasie}.
  \end{itemize}

  \vspace{0.2cm}
  \item Ces composantes sont supposées \textbf{indépendantes}.
  \item La composante additive $a_i$ est la seule transmise en moyenne à la descendance :
  elle est appelée \textit{breeding value}.
\end{itemize}
\end{frame}


\begin{frame}{Hypothèse infinitésimale et variance additive}

\only<1>{
\begin{itemize}
  \item Le caractère est supposé contrôlé par un grand nombre de loci
  ayant chacun un effet faible : \textbf{hypothèse infinitésimale}.

  \vspace{0.2cm}
  \item Sous cette hypothèse :
  \[
  g_i \sim \mathcal{N}(0, \sigma_g^2)
  \]

  \item En particulier, pour la composante additive :
  \[
  a_i \sim \mathcal{N}(0, \sigma_a^2)
  \]
  où $\sigma_a^2$ est la \textbf{variance génétique additive}.
  \end{itemize}
  }
  \only<2>{
\begin{itemize}
  
  \item En écriture multivariée :
  \[
  \boldsymbol{a} \sim \mathcal{N}_I(\boldsymbol{0}, \sigma_a^2 A)
  \]
  \item $A$ est la \textbf{matrice de relations génétiques additives},
  dérivable :
  \begin{itemize}
    \item du pédigrée (espérance mendélienne),
    \item ou directement des données de génotypage.
  \end{itemize}

  \vspace{0.2cm}
  \item Tant que $\sigma_a^2 > 0$, la sélection des individus
  ayant les plus grandes valeurs additives $a_i^{(s)}$
  permet d’augmenter la moyenne génotypique au fil des générations.
\end{itemize}
}
\end{frame}

\subsection{Modélisation statistique}
\begin{frame}{Modélisation statistique}
\begin{itemize}
  \item Les valeurs génotypiques $\{g_i\}$ ne sont \textbf{pas observables directement}.
  \item On dispose uniquement d’observations phénotypiques :
  \[
  \{y_n\}_{1 \le n \le N}, \quad \mathbb{E}[y_n] = \mu_0, \quad \text{Var}[y_n] = \sigma_p^2
  \]

  \vspace{0.2cm}
  \item Hypothèses statistiques de base :
  \begin{itemize}
    \item données indépendantes,
    \item distribution Normale,
    \item absence de covariance entre erreurs.
  \end{itemize}

  \vspace{0.2cm}
  \item La vraisemblance marginale s’écrit alors :
  \[
  y_n \mid \mu_0, \sigma_p^2 \;\overset{\text{i.i.d}}{\sim}\; \mathcal{N}(\mu_0, \sigma_p^2)
  \]

  \item Objectif : utiliser les phénotypes pour \textbf{estimer ou prédire} les effets génétiques.
\end{itemize}
\end{frame}


\begin{frame}{Modélisation statistique}
\begin{itemize}
  \item En pratique, on observe plusieurs répétitions par génotype :
  \[
  N = I \times J
  \]

  \vspace{0.2cm}
  \item Le modèle phénotypique s’écrit :
  \[
  y_{ij} = \mu + g_i + \epsilon_{ij}
  \]
  avec :
  \[
  \epsilon_{ij} \mid \sigma^2 \;\overset{\text{i.i.d}}{\sim}\; \mathcal{N}(0, \sigma^2)
  \]

  \vspace{0.2cm}
  \item Conditionnellement aux effets génétiques :
  \[
  y_{ij} \mid \mu, g_i, \sigma^2
  \sim \mathcal{N}(\mu + g_i, \sigma^2)
  \]

  \item $g_i$ est traité comme un \textbf{effet aléatoire} : on est en présence
  d’un \textbf{modèle linéaire mixte}.
\end{itemize}
\end{frame}



\begin{frame}{Variances et héritabilité}

\begin{itemize}

\only<1>{
  \item Après intégration des effets génétiques :
  \[
  y_{ij} \mid \mu_0, \sigma_g^2, \sigma^2
  \sim \mathcal{N}(\mu_0, \sigma_g^2 + \sigma^2)
  \]

  \item Décomposition de la variance phénotypique :
  \[
  \sigma_p^2 = \sigma_g^2 + \sigma^2
  \]

  \item Les composantes de variance peuvent être estimées par \textit{ReML}.
  \item Les prédictions des effets génétiques $g_i$ reposent sur les équations de Henderson
  (BLUP).
  }
  
  \only<2>{

  \vspace{0.2cm}
  \item Corrélation entre génotype et phénotype :
  \[
  \rho_{g,y} = \frac{\text{Cov}(g,y)}{\sigma_g \sigma_p}
  \]

  \item En l’absence de covariance génotype--environnement :
  \[
  H^2 = \frac{\sigma_g^2}{\sigma_p^2}
  \]

  \item $H^2$ mesure la capacité du dispositif expérimental à révéler les différences génétiques.
}
\end{itemize}
\end{frame}

\begin{frame}{Relation entre parents et descendants}
\begin{itemize}
\only<1>{
  \item On cherche à relier le phénotype des enfants au phénotype de leurs parents.
  \item On considère :
  \begin{itemize}
    \item $y_{\text{mère}}$ : phénotype de la mère,
    \item $y_{\text{père}}$ : phénotype du père,
    \item $y = \dfrac{y_{\text{mère}} + y_{\text{père}}}{2}$ : \textbf{parent moyen},
    \item $y_e$ : phénotype de l’enfant.
  \end{itemize}
}
 \only<2>{
  \item Décomposition des phénotypes parentaux :
  \[
  y_{\text{mère}} = a_{\text{mère}} + d_{\text{mère}} + \zeta_{\text{mère}} + \epsilon_{\text{mère}}
  \]
  \[
  y_{\text{père}} = a_{\text{père}} + d_{\text{père}} + \zeta_{\text{père}} + \epsilon_{\text{père}}
  \]

  \item Le phénotype de l’enfant s’écrit :
  \[
  y_e = a_{\text{mère}} + a_{\text{père}} + \epsilon_e
  \]

  \item Seule la composante \textbf{additive} est transmise à la descendance.
  }
\end{itemize}
\end{frame}


\begin{frame}{Régression enfants--parents}
\begin{itemize}
  \item Pour caractériser la relation parents–enfants, on considère la régression linéaire :
  \[
  y_e = \alpha + \beta_{\text{enfants},\text{parents}} \, y + \eta
  \]

  \item La pente de la droite est donnée par :
  \[
  \beta_{\text{enfants},\text{parents}}
  = \frac{\text{Cov}[y, y_e]}{\text{Var}[y]}
  \]

  \item Hypothèses clés :
  \begin{itemize}
    \item panmixie (accouplements aléatoires),
    \item absence de sélection,
    \item absence de covariance génotype--environnement,
    \item absence de transmission d’effets environnementaux.
  \end{itemize}

  \item Sous ces hypothèses, de nombreux termes de covariance s’annulent.
\end{itemize}
\end{frame}

\begin{frame}{Covariance parents--enfants}
\begin{itemize}
  \item La covariance s’écrit :
  \[
  \text{Cov}[y, y_e]
  = \text{Cov} \left[
  \frac{a_{\text{mère}} + a_{\text{père}}}{2},
  a_{\text{mère}} + a_{\text{père}}
  \right]
  \]

  \item En utilisant l’indépendance entre parents :
  \[
  \text{Cov}[y, y_e]
  = \frac{\text{Var}[a_{\text{mère}}] + \text{Var}[a_{\text{père}}]}{2}
  \]

  \item En panmixie, la variance génétique additive totale est :
  \[
  \sigma_a^2 = \text{Var}[a_{\text{mère}}] + \text{Var}[a_{\text{père}}]
  \]

  \item Ainsi :
  \[
  \text{Cov}[y, y_e] = \frac{\sigma_a^2}{2}
  \]
\end{itemize}
\end{frame}


\begin{frame}{Héritabilité au sens strict}
\begin{itemize}
  \item La variance du parent moyen est :
  \[
  \text{Var}[y]
  = \frac{\text{Var}[y_{\text{mère}}] + \text{Var}[y_{\text{père}}]}{4}
  = \frac{\sigma_p^2}{2}
  \]

  \item La pente de la régression devient :
  \[
  \beta_{\text{enfants},\text{parents}}
  = \frac{\sigma_a^2 / 2}{\sigma_p^2 / 2}
  = \frac{\sigma_a^2}{\sigma_p^2}
  \]

  \item On définit :
  \[
  h^2 = \frac{\sigma_a^2}{\sigma_p^2}
  \quad \text{(héritabilité au sens strict)}
  \]

  \item $h^2$ est aussi la corrélation entre phénotype et valeur génétique additive :
  \[
  \rho_{a,y} = h
  \]
\end{itemize}
\end{frame}

\section{Equation du selectionneur}
\subsection{Differentiel de selection}
\begin{frame}
\frametitle{Differentiel de selection}
 \begin{center}
      \includegraphics[width=0.9\textwidth,height=\textheight,keepaspectratio=true]{figures/s.jpeg}
 \end{center}
 \vspace{-0.3cm}
\[
S = \mu^{(s)} - \mu_0
\]
\end{frame}


\begin{frame}
\frametitle{Intensite de selection}

\begin{itemize}
 \item $S$ est dépend de l'unité de mesure du phénotype.
\end{itemize}
Pour comparer la sélection sur différents caractères, il est donc recommandé de travailler avec une valeur standardisée,
 l'$intensité de sélection$, notée $i$ (à ne pas confondre avec l'indice $i$ du modèle statistique):

\[
i = \frac{S}{\sigma_p}
\]
\end{frame}


\begin{frame}
\frametitle{Intensite de selection}
 \begin{center}
      \includegraphics[width=0.9\textwidth,height=\textheight,keepaspectratio=true]{figures/int.jpeg}
 \end{center}
\end{frame}


\subsection{Réponse à la sélection}
\begin{frame}{Réponse à la sélection : idée centrale}
\begin{itemize}
  \item L’héritabilité au sens strict vérifie :
  \[
  0 \le h^2 \le 1
  \]
  \item La droite de régression des enfants sur les parents est donc
  \textbf{moins pentue que la droite identité}.
  \item Cette observation est à l’origine du terme de
  \textit{régression vers la moyenne} (Galton).

  \vspace{0.2cm}
  \item Notons :
  \begin{itemize}
    \item $\mu_0$ : moyenne phénotypique de la population parentale,
    \item $\mu^{(s)}$ : moyenne des parents sélectionnés,
    \item $\mu_1$ : moyenne phénotypique de la descendance.
  \end{itemize}

  \vspace{0.2cm}
  \item On définit :
  \[
  S = \mu^{(s)} - \mu_0
  \quad \text{(différentiel de sélection)}
  \]
  \[
  R = \mu_1 - \mu_0
  \quad \text{(réponse à la sélection)}
  \]

  \item La sélection des parents induit donc un changement de moyenne
  chez les descendants.
\end{itemize}
\end{frame}


\begin{frame}{Équation du sélectionneur}
\begin{itemize}
  \item La pente de la régression enfants--parents s’écrit :
  \[
  \beta_{\text{enfants,parents}}
  = \frac{\mu_1 - \mu_0}{\mu^{(s)} - \mu_0}
  = \frac{R}{S}
  \]

  \item Or, on a montré précédemment que :
  \[
  \beta_{\text{enfants,parents}} = h^2
  \]

  \item On obtient alors la relation fondamentale :
  \[
  \boxed{R = h^2 \; S}
  \]

  \vspace{0.2cm}
  \item Cette équation combine :
  \begin{itemize}
    \item une information d’hérédité (\(h^2\)),
    \item une information de sélection (\(S\)),
    \item pour prédire le changement intergénérationnel (\(R\)).
  \end{itemize}

  \item Elle est connue sous le nom d’\textbf{équation du sélectionneur}.
\end{itemize}
\end{frame}


\begin{frame}{Interprétation et optimisation de la sélection}
\begin{itemize}
\only<1>{
  \item Le différentiel de sélection peut s’écrire :
  \[
  S = i \, \sigma_p
  \]
  où $i$ est l’intensité de sélection.

  \item La réponse devient :
  \[
  R = i \, h^2 \, \sigma_p
  = i \, h \, \sigma_a
  \]

  \item Le terme $h = \frac{\sigma_a}{\sigma_p}$ est aussi :
  \[
  h = \rho_{a,y}
  \]
  la corrélation entre phénotype observé et valeur génétique additive,
  appelée \textbf{précision} ou \textit{accuracy} ($r$).

  \vspace{0.2cm}
  \item On peut donc écrire :
  \[
  \boxed{R = i \; r \; \sigma_a}
  \]
}

\only<2>{
  \item \textbf{En résumé} :
  \[
  \text{Réponse} =
  \text{intensité} \times \text{précision} \times \text{variance additive}
  \]

  \vspace{0.2cm}
  \item Une sélection trop intense augmente $R$ à court terme,
  mais réduit la variance génétique disponible à long terme.
  }
\end{itemize}
\end{frame}


\section{Effet de la selection sur la variance}

\begin{frame}{Effet de la sélection sur la variance génétique}
\only<1-5>{
\begin{itemize}
\item<1-> \textbf{Question clé :}  
La sélection modifie les fréquences alléliques aux locus impliqués dans la variation d’un trait.  
Peut-on s’attendre à une modification rapide de la variance génétique d’une population ?

\item<2-> Pour un locus à effets additifs :  
Si l’effet de chaque allèle est faible, le changement de variance au locus est très limité sur quelques générations.

\item<3-> Pour un caractère polygénique (somme de plusieurs locus) :  
La sélection crée une covariance négative entre locus non liés.

\item<4-> Cette covariance négative réduit la variance additive totale observée dans la population.  
C’est l’\textbf{effet Bulmer}.

\item<5-> Cette diminution de variance n’est pas définitive :  
elle est stockée dans le \textbf{déséquilibre de liaison} et peut se relâcher si la pression de sélection diminue.
\end{itemize}
}

\only<2->{
\begin{itemize}
\item<6-> En résumé :  
\begin{itemize}
    \item La sélection modifie les fréquences alléliques très progressivement à chaque locus.  
    \item La variance diminue surtout à cause de covariances négatives entre loci.  
    \item La variance totale peut se récupérer avec le temps et le recombinage.
\end{itemize}
\end{itemize}
}
\end{frame}


%%%%%%%%%%%%%%%%%%%%%%%%%%% 
%% -- Activité --%
%%%%%%%%%%%%%%%%%%%%%%%%%%% 
\section{Activité - 45-60 min}
\begin{frame}
 \frametitle{Activité}
%% Donner 3 articles, chaque etudiant choisit un article à son choix
\begin{itemize}
\setbeamertemplate{itemize item}[square]
\setlength{\itemsep}{0.6em}
\item Code R
\end{itemize}
\end{frame}

% After your last numbered slide
\appendix
\newcounter{finalframe}
\setcounter{finalframe}{\value{framenumber}}

\setcounter{framenumber}{\value{finalframe}}

\end{document}

%%% Local Variables:
%%% mode: latex
%%% TeX-master: t
%%% End:
