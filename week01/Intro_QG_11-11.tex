\documentclass[c]{beamer} % use [t] to top-justified body text by default
% \documentclass[c,handout]{beamer}
\usepackage{graphicx}
\usepackage{hyperref}
\hypersetup{colorlinks, linkcolor=black, urlcolor=gray}
\usepackage{amsmath}
\usepackage{bm} % to have mathematical symbols in bold
\usepackage{multirow}
\usepackage{tikz}
% \usepackage[francais]{babel}
\usepackage[english]{babel}
\usepackage[T1]{fontenc} 
\usepackage[utf8]{inputenc}
\usepackage{tcolorbox}
\usepackage{multicol}
\usepackage{xcolor}

% https://tex.stackexchange.com/a/82558
\usepackage[absolute,overlay]{textpos}
% \usepackage[texcoord,grid,gridunit=mm,gridcolor=red!10,subgridcolor=green!10]{eso-pic}

\graphicspath{{./figures/}}

% -----------------------------------------------------------------------------
\setbeamertemplate{navigation symbols}{}
\setbeamercolor{alerted text}{fg=purple}
\setbeamertemplate{caption}[numbered]
\setbeamerfont{caption}{size=\scriptsize}
\setbeameroption{hide notes}
% \setbeameroption{show notes}
\setbeamertemplate{note page}[plain]

\setbeamertemplate{frametitle}{
  \vspace{0.2cm}
  \makebox[\linewidth][c]{
    \parbox{0.9\linewidth}{\centering
      \bfseries\insertframetitle
    }
  }
}

\setbeamertemplate{footline}
{
  \leavevmode
  \hbox{
    \hspace*{-0.06cm}
    \begin{beamercolorbox}[wd=.2\paperwidth,ht=2.25ex,dp=1ex,center]{author in head/foot}
      \usebeamerfont{author in head/foot}\insertshortauthor \hspace*{1em} \insertshortinstitute
    \end{beamercolorbox}
    \begin{beamercolorbox}[wd=.50\paperwidth,ht=2.25ex,dp=1ex,center]{section in head/foot}
      \usebeamerfont{section in head/foot}\insertshorttitle
    \end{beamercolorbox}
    \begin{beamercolorbox}[wd=.27\paperwidth,ht=2.25ex,dp=1ex,right]{section in head/foot}%
      \usebeamerfont{section in head/foot}\insertshortdate{}\hspace*{2em}
      \insertframenumber{} / \inserttotalframenumber\hspace*{2ex}
    \end{beamercolorbox}
  }
  \vskip0pt
}

\AtBeginSection[]
{
  \begin{frame}
    \frametitle{Outline}
    \addtocounter{framenumber}{-1}
    \tableofcontents[currentsection]
  \end{frame}
}

\AtBeginSubsection[]
{
  \begin{frame}
    \frametitle{Outline}
    \addtocounter{framenumber}{-1}
    \tableofcontents[currentsubsection,hideothersubsections,sectionstyle=show/shaded,subsectionstyle=show/shaded]
  \end{frame}
}
% -----------------------------------------------------------------------------


%%%%%%%%%%%%%%%%%%%%%%%%%%%%
%% Beginning of the document
%%%%%%%%%%%%%%%%%%%%%%%%%%%%
\begin{document}

\title[Génét. Quanti., UniQ ]{Cours 1 – Introduction à la génétique quantitative}
\author[J. Salomon]{Jemay Salomon}
\institute[]{\small UMR GQE Le Moulon \\ Université Paris–Saclay, INRAE, CNRS, AgroParisTech}
\date{Nov. 11, 2025}

%% Title page
\begin{frame}[plain,t]
\noindent
%\includegraphics[height=0.7cm]{figures/logo-gqe-le-moulon.png}\hfill
%\includegraphics[height=0.8cm]{figures/logos_mobidiv.jpeg}

\vfill
\begin{center}
  %PLANTCOM meeting, Dijon

  \bigskip

  \bigskip
  
  {\fontsize{14pt}{16pt}\selectfont\bfseries \inserttitle}\\[0.8cm] 
  {\large\insertauthor}\\[0.4cm] 
  {\normalsize\insertinstitute}
\end{center}
\vfill

\noindent
\includegraphics[width=2cm]{figures/logo_faculte_sciences.png}\hfill
\includegraphics[width=0.8cm]{figures/logo-inrae-fond-blanc.png}\hfill
\includegraphics[width=0.5cm]{figures/logo_CNRS_biologie.png}\hfill
\includegraphics[width=1.5cm]{figures/logo_agroparistech.png}
\end{frame}



\begin{frame}
  \frametitle{Outline}
  \tableofcontents
\end{frame}


%%%%%%%%%%%%%%%%%%%%%%%%%%% 
%% --Begin-Document --%
%%%%%%%%%%%%%%%%%%%%%%%%%%% 
\section{Contexte et motivation}

\begin{frame}
  \frametitle{Population}
  \begin{center}
    \begin{figure}
    \begin{overprint}
    
 \onslide<1>\includegraphics[width=0.95\textwidth,height=0.90\textheight,keepaspectratio=true]{figures/motivation_1.png}%
 
        \end{overprint}
    \end{figure}
  \end{center}
\end{frame}


\begin{frame}
  \frametitle{Variation phénotypique}
  \begin{center}
    \begin{figure}
    \begin{overprint}
    
 \includegraphics[width=0.95\textwidth,height=0.90\textheight,keepaspectratio=true]{figures/motivation_2.png}%
 
        \end{overprint}
    \end{figure}
  \end{center}
\end{frame}

\begin{frame}
  \frametitle{Variation phénotypique}
  \centering
  \begin{tikzpicture}
    % Image in background
    \node[inner sep=0pt, opacity=0.8] at (0,0)
      {\includegraphics[width=0.95\textwidth,height=0.90\textheight,keepaspectratio]{figures/motivation_2.png}};
    % Text in foreground (on top)
    \node[align=center, text=black, font=\bfseries\Large] at (0,0)
      {Gènes, Environnement,\\Gènes × Environnement};
  \end{tikzpicture}
\end{frame}


\begin{frame}
  \frametitle{Consequences}
  \begin{center}
    
 \includegraphics[width=0.75\textwidth,height=0.75\textheight,keepaspectratio=true]{figures/yield.png}%
  {\footnotesize \newline cf-Principles of Plant Genetics and Breeding (2012)}
  
  \end{center}
\end{frame}

%%%%%%%%%%%%%%%%%%%%%%%%%%% 
%% --section --%
%%%%%%%%%%%%%%%%%%%%%%%%%%% 
\section{Concept theorique}

\subsection{Définition}

\begin{frame}
 \frametitle{Déf.}
 \parbox[t]{0.95\linewidth}{
 \textbf {La génétique quantitative vise à relier la variation des traits complexes à leur base génétique, 
 pour mieux comprendre et prédire l’architecture génétique et    l’évolution des populations sur le long terme.}}
\end{frame}



\begin{frame}
\frametitle{Déf.}

\begin{itemize}
    \item<1-> \textbf{Génétique quantitative classique}
    \begin{itemize}
    \setlength{\itemsep}{0.4em}
        \item<2-> S'intéresse à la variation phénotypique globale et à sa base génétique, en considérant l'ensemble des gènes.
        \item<3-> Analyse holistique de tous les gènes, sans distinguer contributions majeures ou mineures.
        \item<4-> Fournit une vision globale de la génétique des traits complexes.
    \end{itemize}
\vspace{0.8cm}
    \item<5-> \textbf{Génétique quantitative moléculaire}
    \begin{itemize}
    \setlength{\itemsep}{0.4em}
        \item<6-> Se concentre sur l'association entre les sites d'ADN polymorphes et les variations phénotypiques.
        \item<7-> Analyse détaillée de l'architecture génétique : gènes majeurs (analyse ciblée) et gènes mineurs (vue globale).
        \item<8-> Permet de comprendre comment des loci spécifiques contribuent aux variations, utile pour la sélection ou la prédiction génétique.
    \end{itemize}
\end{itemize}

\end{frame}


%%%%%%%%%%%%%%%%%%%%%%%%%%% 
%% --section --%
%%%%%%%%%%%%%%%%%%%%%%%%%%% 
\subsection{Caractères quantitatifs versus caractères qualitatifs}
\begin{frame}
  \frametitle{Distribution des caractères quantitatifs et qualitatifs}

  % ===============================
  % ---- Partie 1 : Caractère quantitatif --
  % ===============================
  \only<1-5>{
    \begin{columns}[T]
      \hspace{0.2cm}
      
      % Colonne gauche : image
      \begin{column}{0.70\textwidth} 
        \includegraphics[width=\textwidth,height=\textheight,keepaspectratio]{figures/quanti.png}
      \end{column}
      
      \hspace{-0.4cm}
      % Colonne droite : texte
      \begin{column}{0.47\textwidth} 
        \small
        \setbeamertemplate{itemize item}[ball]
        \begin{itemize}
          \item<2-> \parbox[t]{0.95\linewidth}{
            Variation continue
          }
          \vspace{0.3cm}

          \item<3-> \parbox[t]{0.95\linewidth}{
           Sous contrôle de nombreux gènes
          }
          \vspace{0.3cm}

          \item<4-> \parbox[t]{0.95\linewidth}{
            Influencer largement par l'environnement
            \\ ...
          }
           \vspace{0.3cm}
          
          \item<5-> \parbox[t]{0.95\linewidth}{
            Ex. Hauteur des plantes, diamètre tige, etc...
          }
        \end{itemize}
      \end{column}
    \end{columns}
  }


  % ===============================
  % ---- Partie 2 : Caractère qualitatif ----
  % ===============================
  \only<5-9>{
    \begin{columns}[T]
      \hspace{0.2cm}
      
      % Colonne gauche : image
      \begin{column}{0.70\textwidth} 
        \includegraphics[width=\textwidth,height=\textheight,keepaspectratio]{figures/quali.png}
      \end{column}
      
      \hspace{-0.4cm}
      % Colonne droite : texte
      \begin{column}{0.47\textwidth} 
        \small
        \setbeamertemplate{itemize item}[ball]
        \begin{itemize}
          \item<6-> \parbox[t]{0.95\linewidth}{
           Variation discontinue
          }
          \vspace{0.3cm}

          \item<7-> \parbox[t]{0.95\linewidth}{
            Sous controle d'un ou peu de gènes
          }
          \vspace{0.3cm}

          \item<8-> \parbox[t]{0.95\linewidth}{
          Peu ou pas d'influence de l'environnement
          \\...
          }
          
           \vspace{0.3cm}
              \item<9-> \parbox[t]{0.95\linewidth}{
          Ex. Couleur des fleurs, résistance/sensibilité, etc..
          }
        \end{itemize}
      \end{column}
    \end{columns}
  }
\end{frame}



%%%%%%%%%%%%%%%%%%%%%%%%%%% 
%% --section --%
%%%%%%%%%%%%%%%%%%%%%%%%%%% 

\section{Concept statistique}
\subsection{Modèle de base}

\begin{frame}
 \frametitle{Rappel}
 
 %\begin{itemize}
 % ADN
 % Genes
 % Alleles
 % Effet allélique
 % Locus
 % Dominance
 % Recessivité
 % Epistasie
 % Génome
 % Background génétique
 %\end{itemize}
\end{frame}

\begin{frame}
  \frametitle{Décomposition de la variation phénotypique (1/2)}
  Pour un génotype dans un environnement donné on a :
  \begin{itemize}
    \item<2-> P : Phénotype
    \item<3-> $\mu$ : moyenne de la population
    \item<4-> G : Valeur génétique
    \item<5-> E : Environnement (macro $\neq$ micro, spécifique)
  \end{itemize}

  \only<6->{
    \begin{align*}
      P &= \mu + G + E
    \end{align*}
  }

\end{frame}

\begin{frame}
  \frametitle{Décomposition de la variation phénotypique (2/2)}
  
  Pour un génotype dans un environnement donné on a :
  \only<1->{
    \[
      G = [g_1 + g_i + g_n]
    \]
  }
  
  \only<2->{
    \begin{itemize}
      \item<2-> P : Phénotype
      \item<2-> $\mu$ : moyenne de la population
      \item<2-> G = $[g_1 + g_i + g_n]$ : Valeur génétique
      \item<2-> E : Environnement (macro $\neq$ micro, spécifique)
    \end{itemize}
  }
  
  \only<3->{
    \[
      P = \mu + [g_1 + g_i + g_n] + E
    \]
  }

\end{frame}


\begin{frame}
  \frametitle{Hértitabilité}
\end{frame}

%%%%%%%%%%%%%%%%%%%%%%%%%%% 
%% --section --%
%%%%%%%%%%%%%%%%%%%%%%%%%%% 


\section{Application à l'amélioration des plantes}
\begin{frame}
\end{frame}


%%%%%%%%%%%%%%%%%%%%%%%%%%% 
%% -- Activité --%
%%%%%%%%%%%%%%%%%%%%%%%%%%% 
\section{Activité - 45 min}
\begin{frame}
%% Donner 3 articles, chaque etudiant choisit un article à son choix
\begin{itemize}
\setbeamertemplate{itemize item}[square]
\setlength{\itemsep}{0.6em}
\item Choisir un article
\item Donner le contexte/problématique
\item Decrire le modèle et les différents termes du modèle utilisé
\item Decrire 2 parmi les résultats obtenus (tableau ou graphique)
\item Donner votre opinion, force, faiblesse, des changements vous aurez faits si vous devez refaire cette étude
\item A envoyer: jemay.salomon@inrae.fr (45 min)
\item Max: 3 pages/en pdf
\end{itemize}
\end{frame}

% After your last numbered slide
\appendix
\newcounter{finalframe}
\setcounter{finalframe}{\value{framenumber}}

\setcounter{framenumber}{\value{finalframe}}

\end{document}

%%% Local Variables:
%%% mode: latex
%%% TeX-master: t
%%% End:
