\documentclass[c]{beamer} % use [t] to top-justified body text by default
% \documentclass[c,handout]{beamer}
\usepackage{graphicx}
\usepackage{hyperref}
\hypersetup{colorlinks, linkcolor=black, urlcolor=gray}
\usepackage{amsmath}
\usepackage{bm} % to have mathematical symbols in bold
\usepackage{multirow}
\usepackage{tikz}
% \usepackage[francais]{babel}
\usepackage[english]{babel}
\usepackage[T1]{fontenc} 
\usepackage[utf8]{inputenc}
\usepackage{tcolorbox}
\usepackage{multicol}
\usepackage{xcolor}

% https://tex.stackexchange.com/a/82558
\usepackage[absolute,overlay]{textpos}
% \usepackage[texcoord,grid,gridunit=mm,gridcolor=red!10,subgridcolor=green!10]{eso-pic}

\graphicspath{{./figures/}}

% -----------------------------------------------------------------------------
\setbeamertemplate{navigation symbols}{}
\setbeamercolor{alerted text}{fg=purple}
\setbeamertemplate{caption}[numbered]
\setbeamerfont{caption}{size=\scriptsize}
\setbeameroption{hide notes}
% \setbeameroption{show notes}
\setbeamertemplate{note page}[plain]

\setbeamertemplate{frametitle}{
  \vspace{0.2cm}
  \makebox[\linewidth][c]{
    \parbox{0.9\linewidth}{\centering
      \bfseries\insertframetitle
    }
  }
}

\setbeamertemplate{footline}
{
  \leavevmode
  \hbox{
    \hspace*{-0.06cm}
    \begin{beamercolorbox}[wd=.2\paperwidth,ht=2.25ex,dp=1ex,center]{author in head/foot}
      \usebeamerfont{author in head/foot}\insertshortauthor \hspace*{1em} \insertshortinstitute
    \end{beamercolorbox}
    \begin{beamercolorbox}[wd=.50\paperwidth,ht=2.25ex,dp=1ex,center]{section in head/foot}
      \usebeamerfont{section in head/foot}\insertshorttitle
    \end{beamercolorbox}
    \begin{beamercolorbox}[wd=.27\paperwidth,ht=2.25ex,dp=1ex,right]{section in head/foot}%
      \usebeamerfont{section in head/foot}\insertshortdate{}\hspace*{2em}
      \insertframenumber{} / \inserttotalframenumber\hspace*{2ex}
    \end{beamercolorbox}
  }
  \vskip0pt
}

\AtBeginSection[]
{
  \begin{frame}
    \frametitle{Outline}
    \addtocounter{framenumber}{-1}
    \tableofcontents[currentsection]
  \end{frame}
}

\AtBeginSubsection[]
{
  \begin{frame}
    \frametitle{Outline}
    \addtocounter{framenumber}{-1}
    \tableofcontents[currentsubsection,hideothersubsections,sectionstyle=show/shaded,subsectionstyle=show/shaded]
  \end{frame}
}
% -----------------------------------------------------------------------------


%%%%%%%%%%%%%%%%%%%%%%%%%%%%
%% Beginning of the document
%%%%%%%%%%%%%%%%%%%%%%%%%%%%
\begin{document}

\title[Génét. Quanti., UniQ ]{Cours 2 – Le Modèle génétique et le cas biallélique}
\author[J. Salomon]{Jemay Salomon}
\institute[]{\small UMR GQE Le Moulon \\ Université Paris–Saclay, INRAE, CNRS, AgroParisTech}
\date{Nov. 25, 2025}

%% Title page
\begin{frame}[plain,t]
\noindent
%\includegraphics[height=0.7cm]{figures/logo-gqe-le-moulon.png}\hfill
%\includegraphics[height=0.8cm]{figures/logos_mobidiv.jpeg}

\vfill
\begin{center}
  %PLANTCOM meeting, Dijon

  \bigskip

  \bigskip
  
  {\fontsize{14pt}{16pt}\selectfont\bfseries \inserttitle}\\[0.8cm] 
  {\large\insertauthor}\\[0.4cm] 
  {\normalsize\insertinstitute}
\end{center}
\vfill

\noindent
\includegraphics[width=2cm]{figures/logo_faculte_sciences.png}\hfill
\includegraphics[width=0.8cm]{figures/logo-inrae-fond-blanc.png}\hfill
\includegraphics[width=0.5cm]{figures/logo_CNRS_biologie.png}\hfill
\includegraphics[width=1.5cm]{figures/logo_agroparistech.png}
\end{frame}



\begin{frame}
  \frametitle{Outline}
  \tableofcontents
\end{frame}


%%%%%%%%%%%%%%%%%%%%%%%%%%% 
%% --Begin-Document --%
%%%%%%%%%%%%%%%%%%%%%%%%%%% 
\section{Rappel}
\begin{frame}
  \frametitle{Rappel des concepts génétiques}

  % ADN
  \only<1>{
    \textbf{ADN :} support de l'information génétique.\\[0.3cm]
    \begin{center}
      \includegraphics[width=0.75\textwidth,height=0.75\textheight,keepaspectratio=true]{figures/ADN_double_helix.jpg}
    \end{center}
  }

  % Gène
  \only<2>{
    \textbf{Gène :} unité fonctionnelle d'hérédité.\\[0.3cm]
    \begin{center}
      \includegraphics[width=0.75\textwidth,height=0.75\textheight,keepaspectratio=true]{figures/gene_structure_simple.png}
    \end{center}
  }

  % Allèle
  \only<3>{
    \textbf{Allèle :} version alternative d'un même gène.\\[0.3cm]
    \begin{center}
      \includegraphics[width=0.75\textwidth,height=0.75\textheight,keepaspectratio=true]{figures/allele_graphic.png}
    \end{center}
  }

  % Locus
  \only<4>{
    \textbf{Locus :} position d'un gène ou allèle sur le génome.\\[0.3cm]
    \begin{center}
      \includegraphics[width=0.75\textwidth,height=0.75\textheight,keepaspectratio=true]{figures/locus_diagram.png}
    \end{center}
  }

  % Dominance
  \only<5>{
    \textbf{Dominance :} interaction entre allèles d’un même gène.\\[0.3cm]
    \begin{center}
      \includegraphics[width=0.75\textwidth,height=0.75\textheight,keepaspectratio=true]{figures/dominance_recessive.jpg}
    \end{center}
  }

  % Épistasie
  \only<6>{
    \textbf{Épistasie :} interaction entre gènes situés à des loci différents.\\[0.3cm]
    \begin{center}
      \includegraphics[width=0.75\textwidth,height=0.75\textheight,keepaspectratio=true]{figures/epistasis.jpg}
    \end{center}
  }

\end{frame}


%%%%%%%%%%%%%%%%%%%%%%%%%%% 
%% --section --%
%%%%%%%%%%%%%%%%%%%%%%%%%%% 
\section{Le modèle à un locus}
% Slide 1 : décomposition phénotypique
\begin{frame}
  \frametitle{Décomposition de la valeur génétique (G)}

  \textbf{Population :} panmictique, taille infinie, reproduction sexuée.\\[0.15cm]
  \textbf{Un seul locus} $A$ influence la valeur génétique.\\
  Génotypes : $A_iA_j$ avec $A_iA_j = A_jA_i$.

  \vspace{0.3cm}
  \begin{itemize}
    \item<2-> $P$ : Phénotype
    \item<3-> $\mu$ : Moyenne de la population
    \item<4-> $G$ : Valeur génétique
    \item<5-> $E$ : Effet de l'environnement
  \end{itemize}

  \only<6->{
    \[
      P = \mu + G + E
    \]
  }

\end{frame}

% Slide 2 : modèle génétique de Fisher
\begin{frame}
  \frametitle{Décomposition de la valeur génétique (G)}

  Valeur génétique d'un génotype $A_iA_j$ :
  \[
    G_{ij} = \alpha_i + \alpha_j + B_{ij}
  \]

\[
\begin{array}{r l c}
  G = & \text{effet moyen du gène paternel} & + \\
      & \text{effet moyen du gène maternel} & + \\
      & \text{effet d'interaction entre (dominance)} & 
\end{array}
\]

  \vspace{0.3cm}
  où :
  \begin{itemize}
    \item<2-> $\alpha_i$ : effetadditif de l’allèle $A_i$
    \item<3-> $\alpha_j$ : effet additif de l’allèle $A_j$
    \item<4-> $B_{ij}$ : déviation due à la dominance (interaction $A_iA_j$)
  \end{itemize}

\end{frame}


\begin{frame}
\frametitle{Effet moyen associé à un allèle (\(\alpha_i\))}

L'effet moyen de l'allèle \(A_i\) (average effect, Fisher) est défini comme :

\[
\alpha_i = E_j(G_{ij}) = \sum_j p_j G_{ij}
\]

\begin{itemize}
  \item \(p_j\) : fréquence de l'allèle \(A_j\)
  \item Panmixie : \(j\) est tiré indépendamment de \(i\)
  \item \(\alpha_i\) a par construction une \textbf{moyenne nulle} dans la population
\end{itemize}
\end{frame}

\begin{frame}
\frametitle{Effet de dominance (\(\beta_{ij}\))}

L'effet de dominance est défini comme l'interaction entre les deux allèles :

\[
\beta_{ij} = G_{ij} - \alpha_i - \alpha_j
\]

\begin{itemize}
  \item Moyenne nulle : \(E_j(\beta_{ij}) = 0\)
  \item Par construction, \(\alpha_i\) et \(\beta_{ij}\) sont \textbf{indépendants}
  \item Covariances nulles entre effets tirés indépendamment :
  \[
  \text{cov}(\alpha_i, \beta_{ij}) = 0, \quad 
  \text{cov}(\beta_{ij}, \beta_{ik}) = 0
  \]
\end{itemize}
\end{frame}

\begin{frame}
\frametitle{Indépendance des effets alléliques}

Sous panmixie, les deux allèles \(i\) et \(j\) d'un individu sont tirés indépendamment :

\[
\text{cov}(\alpha_i, \alpha_j) = 0
\]

De même : 
\[
E(\alpha_j) = E(\beta_{ij}) = 0
\]
\end{frame}

\begin{frame}
\frametitle{Variance génétique}

La variance génétique d'un génotype est :

\[
V_G = V(G_{ij}) = V(\alpha_i) + V(\alpha_j) + V(\beta_{ij}) \quad (\text{covariances nulles})
\]

\[
V_G = 2 V(\alpha) + V(\beta)
\]

On définit :
\[
\begin{aligned}
V_A &= 2 V(\alpha) = 2 E(\alpha_i^2) \quad \text{(variance additive)}\\
V_D &= V(\beta) = E(\beta_{ij}^2) \quad \text{(variance de dominance)}
\end{aligned}
\]

Donc :
\[
V_G = V_A + V_D
\]
\end{frame}


\begin{frame}
\frametitle{Héritabilité au sens strict}

L'héritabilité au sens strict (\(h^2_{ss}\)) est la proportion de la variance phénotypique due à la \textbf{variance génétique additive} :

\[
h^2_{ss} = \frac{V_A}{V_P}
\]

où \(V_P\) est la variance phénotypique totale.
\end{frame}






%%%%%%%%%%%%%%%%%%%%%%%%%%% 
%% --section --%
%%%%%%%%%%%%%%%%%%%%%%%%%%% 
\section{Généralisation du modèle}



%%%%%%%%%%%%%%%%%%%%%%%%%%% 
%% --section --%
%%%%%%%%%%%%%%%%%%%%%%%%%%% 
\section{Conséquences pour la pratique de l’amélioration génétique}


%%%%%%%%%%%%%%%%%%%%%%%%%%% 
%% -- Activité --%
%%%%%%%%%%%%%%%%%%%%%%%%%%% 
\section{Activité - 45 min}
\begin{frame}
 \frametitle{Activité}
 
\begin{itemize}
\setbeamertemplate{itemize item}[square]
\setlength{\itemsep}{0.6em}
\item Code R
\end{itemize}

\end{frame}

% After your last numbered slide
\appendix
\newcounter{finalframe}
\setcounter{finalframe}{\value{framenumber}}

\setcounter{framenumber}{\value{finalframe}}

\end{document}

%%% Local Variables:
%%% mode: latex
%%% TeX-master: t
%%% End:
